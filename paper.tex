% Template for PLoS
% Version 3.4 January 2017
%
% % % % % % % % % % % % % % % % % % % % % %
%
% -- IMPORTANT NOTE
%
% This template contains comments intended
% to minimize problems and delays during our production
% process. Please follow the template instructions
% whenever possible.
%
% % % % % % % % % % % % % % % % % % % % % % %
%
% Once your paper is accepted for publication,
% PLEASE REMOVE ALL TRACKED CHANGES in this file
% and leave only the final text of your manuscript.
% PLOS recommends the use of latexdiff to track changes during review, as this will help to maintain a clean tex file.
% Visit https://www.ctan.org/pkg/latexdiff?lang=en for info or contact us at latex@plos.org.
%
%
% There are no restrictions on package use within the LaTeX files except that
% no packages listed in the template may be deleted.
%
% Please do not include colors or graphics in the text.
%
% The manuscript LaTeX source should be contained within a single file (do not use \input, \externaldocument, or similar commands).
%
% % % % % % % % % % % % % % % % % % % % % % %
%
% -- FIGURES AND TABLES
%
% Please include tables/figure captions directly after the paragraph where they are first cited in the text.
%
% DO NOT INCLUDE GRAPHICS IN YOUR MANUSCRIPT
% - Figures should be uploaded separately from your manuscript file.
% - Figures generated using LaTeX should be extracted and removed from the PDF before submission.
% - Figures containing multiple panels/subfigures must be combined into one image file before submission.
% For figure citations, please use "Fig" instead of "Figure".
% See http://journals.plos.org/plosone/s/figures for PLOS figure guidelines.
%
% Tables should be cell-based and may not contain:
% - spacing/line breaks within cells to alter layout or alignment
% - do not nest tabular environments (no tabular environments within tabular environments)
% - no graphics or colored text (cell background color/shading OK)
% See http://journals.plos.org/plosone/s/tables for table guidelines.
%
% For tables that exceed the width of the text column, use the adjustwidth environment as illustrated in the example table in text below.
%
% % % % % % % % % % % % % % % % % % % % % % % %
%
% -- EQUATIONS, MATH SYMBOLS, SUBSCRIPTS, AND SUPERSCRIPTS
%
% IMPORTANT
% Below are a few tips to help format your equations and other special characters according to our specifications. For more tips to help reduce the possibility of formatting errors during conversion, please see our LaTeX guidelines at http://journals.plos.org/plosone/s/latex
%
% For inline equations, please be sure to include all portions of an equation in the math environment.  For example, x$^2$ is incorrect; this should be formatted as $x^2$ (or $\mathrm{x}^2$ if the romanized font is desired).
%
% Do not include text that is not math in the math environment. For example, CO2 should be written as CO\textsubscript{2} instead of CO$_2$.
%
% Please add line breaks to long display equations when possible in order to fit size of the column.
%
% For inline equations, please do not include punctuation (commas, etc) within the math environment unless this is part of the equation.
%
% When adding superscript or subscripts outside of brackets/braces, please group using {}.  For example, change "[U(D,E,\gamma)]^2" to "{[U(D,E,\gamma)]}^2".
%
% Do not use \cal for caligraphic font.  Instead, use \mathcal{}
%
% % % % % % % % % % % % % % % % % % % % % % % %
%
% Please contact latex@plos.org with any questions.
%
% % % % % % % % % % % % % % % % % % % % % % % %

\documentclass[10pt,letterpaper]{article}
\usepackage[top=0.85in,left=2.75in,footskip=0.75in]{geometry}

% amsmath and amssymb packages, useful for mathematical formulas and symbols
\usepackage{amsmath,amssymb}

% Use adjustwidth environment to exceed column width (see example table in text)
\usepackage{changepage}

% Use Unicode characters when possible
\usepackage[utf8x]{inputenc}

% textcomp package and marvosym package for additional characters
\usepackage{textcomp,marvosym}

% cite package, to clean up citations in the main text. Do not remove.
\usepackage{cite}

% Use nameref to cite supporting information files (see Supporting Information section for more info)
% \usepackage{nameref,hyperref}

% line numbers
\usepackage[right]{lineno}

% ligatures disabled
\usepackage{microtype}
\DisableLigatures[f]{encoding = *, family = * }

% color can be used to apply background shading to table cells only
\usepackage[table]{xcolor}

% array package and thick rules for tables
\usepackage{array}

% create "+" rule type for thick vertical lines
\newcolumntype{+}{!{\vrule width 2pt}}

% frames around boxes
\usepackage{mdframed}

% create \thickcline for thick horizontal lines of variable length
\newlength\savedwidth
\newcommand\thickcline[1]{%
  \noalign{\global\savedwidth\arrayrulewidth\global\arrayrulewidth 2pt}%
  \cline{#1}%
  \noalign{\vskip\arrayrulewidth}%
  \noalign{\global\arrayrulewidth\savedwidth}%
}

% \thickhline command for thick horizontal lines that span the table
\newcommand\thickhline{\noalign{\global\savedwidth\arrayrulewidth\global\arrayrulewidth 2pt}%
\hline
\noalign{\global\arrayrulewidth\savedwidth}}


% Remove comment for double spacing
%\usepackage{setspace}
%\doublespacing

% Text layout
\raggedright
\setlength{\parindent}{0.5cm}
\textwidth 5.25in
\textheight 8.75in

% Bold the 'Figure #' in the caption and separate it from the title/caption with a period
% Captions will be left justified
\usepackage[aboveskip=1pt,labelfont=bf,labelsep=period,justification=raggedright,singlelinecheck=off]{caption}
\renewcommand{\figurename}{Fig}

% Use the PLoS provided BiBTeX style
\bibliographystyle{plos2015}

% Remove brackets from numbering in List of References
\makeatletter
\renewcommand{\@biblabel}[1]{\quad#1.}
\makeatother

% Leave date blank
\date{}

% Header and Footer with logo
\usepackage{lastpage,fancyhdr,graphicx}
\usepackage{epstopdf}
\pagestyle{myheadings}
\pagestyle{fancy}
\fancyhf{}
\setlength{\headheight}{27.023pt}
% \lhead{\includegraphics[width=2.0in]{PLOS-submission.eps}}
\rfoot{\thepage/\pageref{LastPage}}
\renewcommand{\footrule}{\hrule height 2pt \vspace{2mm}}
\fancyheadoffset[L]{2.25in}
\fancyfootoffset[L]{2.25in}
% \lfoot{\sf PLOS}

%% Include all macros below

\newcommand{\rulemajor}[2]{\section{#1}\label{#2}}
\newcommand{\ruleref}[1]{Rule~\ref{#1}}
\newcommand{\url}[1]{\texttt{#1}}

%% END MACROS SECTION


\begin{document}
\vspace*{0.2in}

% Title must be 250 characters or less.
\begin{flushleft}
{\Large
\textbf\newline{Ten simple rules for teaching programming} % Please use "sentence case" for title and headings (capitalize only the first word in a title (or heading), the first word in a subtitle (or subheading), and any proper nouns).
}
\newline
% Insert author names, affiliations and corresponding author email (do not include titles, positions, or degrees).
\\
{Neil C.C.\ Brown}\textsuperscript{1{\ddag}*},
{Greg~Wilson}\textsuperscript{2{\ddag}*}
\\
\bigskip
\textbf{1} King's College London / neil.c.c.brown@kcl.ac.uk \\
\textbf{2} Rangle.io / gvwilson@third-bit.com \\

\bigskip

% Insert additional author notes using the symbols described below. Insert symbol callouts after author names as necessary.
%
% Remove or comment out the author notes below if they aren't used.
%
% Primary Equal Contribution Note
{\ddag} These authors contributed equally to this work.

% Additional Equal Contribution Note
% Also use this double-dagger symbol for special authorship notes, such as senior authorship.
% \ddag These authors also contributed equally to this work.

% Current address notes
% \textcurrency Current Address: Dept/Program/Center, Institution Name, City, State, Country % change symbol to "\textcurrency a" if more than one current address note
% \textcurrency b Insert second current address
% \textcurrency c Insert third current address

% Deceased author note
% \dag Deceased

% Group/Consortium Author Note
% \textpilcrow Membership list can be found in the Acknowledgments section.

% Use the asterisk to denote corresponding authorship and provide email address in note below.
* corresponding author

\end{flushleft}

% Please keep the abstract below 300 words

% \linenumbers

% \section*{Abstract}
% FIXME write abstrat

% Please keep the Author Summary between 150 and 200 words
% Use first person. PLOS ONE authors please skip this step.
% Author Summary not valid for PLOS ONE submissions.
% \section*{Author summary}
% FIXME write author summary

\section*{Introduction}

Research from educational psychology suggests that teaching and learning are subject-specific
activities~\cite{mayer-subject}: learning programming has a different set of challenges and techniques
than learning physics or learning to read and write.  Computing is a younger discipline than
mathematics, physics, or biology, and while there have been correspondingly fewer studies of how best to teach it,
there is a growing body of evidence about what works and what doesn't.
This paper presents ten simple rules that should be the foundation of any teaching of programming, whether
formal or informal.

\rulemajor{There is no geek gene}{gene}

Guzdial~\cite{guzdial-myths} refers to the belief that some people are born programmers and others aren't
as ``computing's most enduring and damaging myth.''
This is often ``confirmed'' by looking at university grade distributions,
which are commonly held to be bimodal:
a low-scoring hump of those that will never get it
and a high-scoring hump of those that have the right stuff.
Our first and most important rule is that this is wrong:
competence at programming is not innate,
but is rather a learned skill that can be acquired and improved with practice.

The most powerful evidence for this comes from Patitsas et al.~\cite{patitsas-bimodal}.
They examined grade distributions in introductory computing courses at a large university,
and found that only 5.8\% were actually multi-modal.
More damningly,
they found that computer science faculty were more likely to see distributions as bimodal
if they thought those grades came from a programming class
than if they believed the grades came from some other kind of class,
and that those faculty were even more likely to see the distributions as bimodal
if they believed that some students are innately predisposed to do well in computer science.

Beliefs such as this are known to have powerful effects on education outcomes
\cite{alvidrez-weinstein,brophy,jussim-eccles}.
If instructors believe that ``some kids get it and some kids don't'',
they will (consciously or unconsciously) invest less in those whom they put in the second category.
When combined with cultural stereotypes about who is and isn't a ``natural programmer'',
the downward spiral of under-achievement that results from differential attention
may be partly responsible for the gender imbalance in computing.

\rulemajor{Use peer instruction}{peer-instruction}

One-on-one tutoring is perhaps the ideal form of teaching:
all of a teacher's attention can be focused on one student,
and they can completely customise their teaching for that person
and tailor individual feedback and corrections based on two-way dialogue with them.
In realistic settings,
however,
one teacher must usually teach several, tens, or even hundreds of students at once.
How can teachers possibly hope to clear up many learners' different misconceptions
in these larger settings in a reasonable time?

The best method developed so far for larger-scale classrooms is called Peer Instruction.
Originally created by Eric Mazur at Harvard~\cite{mazur-peer-instruction},
it has been studied extensively in a wide variety of contexts,
including programming \cite{porter-multi,porter-success}.
In simplified form,
peer instruction proceeds in three phases:

\begin{enumerate}

\item The instructor gives learners a brief introduction to the topic.

\item The instructor then gives learners a multiple choice question
  that probes for misconceptions rather than simple factual recall.
  A programming example is given in Figure~\ref{fig:mcq}
  which relates to integer comparison and loops.  The multiple choice question must be well designed.
  There is no point asking a trivial question that all students will get right or one with meaningless wrong answers which no student will pick.
  The ideal questions are those where 40--60\% of students are likely to get the right answer first time~\cite[p23]{nrc-reaching}, and where every wrong answer corresponds to a misconception that will cause it to be picked by at least some students.

\item Learners then vote on the answer to the question.

  \begin{itemize}
  \item If they all have the right answer, the instructor can move on.
  \item If they all have the same wrong answer,
    the instructor addresses that specific misconception.
  \item If they have a mix of right and wrong answers,
    they are given several minutes to discuss those answers with one another
    in small groups (typically 2-4 students)
    and then reconvene and vote again.
  \end{itemize}

\end{enumerate}

Peer instruction is essentially
a way to provide one-to-one mentorship in a scalable way.
Group discussion significantly improves learners' understanding
because it forces them to clarify their thinking,
which can be enough to call out gaps in reasoning.
Re-polling the class then lets the instructor know if they can move on,
or if further explanation is necessary.
While it significantly outperforms lecture-based instruction in most situations,
it can be problematic if ability levels differ widely
(as they often do in introductory programming classes
because of varied prior experience).
Pair programming (\ruleref{pair-programming}) can be used to mitigate this.

\rulemajor{Use live coding}{live-coding}

Rather than using slides,
instructors should create programs in front of their learners~\cite{rubin-live-coding}.
This is more effective for multiple reasons:

\begin{enumerate}

\item It enables instructors to be more responsive to ``what if?''
  questions. Where a slide deck is like a highway, live coding
  allows instructors to go off road and follow their learners'
  interests or answer unanticipated questions.

\item It facilitates unintended knowledge transfer: students learn
  more than the instructor consciously intends to teach by watching
  \textit{how} instructors do things.  The extra knowledge may be
  high-level (e.g., whether a program is written top-down or
  bottom-up) or fairly low-level (e.g., learning useful editor shortcuts).

\item It slows the instructor down: if the instructor has to type in
  the program as they go along, they can only go twice as fast as
  their learners, rather than ten-fold faster as they could with
  slides -- which risks leaving everyone behind.

\item Learners get to see how instructors diagnose and correct
  mistakes. Novices are going to spend most of their time doing
  this, but it's left out of most textbooks.

\item Watching instructors make mistakes shows learners that it's
  alright to make mistakes of their own~\cite{barker-live-coding}.
  Most people model the behavior of their teachers: if the instructor
  isn't embarrassed about making and talking about mistakes, learners
  will be more comfortable doing so too.

\end{enumerate}

Live coding does have some drawbacks, but with practice, these can be
avoided or worked around:

\begin{enumerate}

\item Instructors can go too slowly, either because they are not good
  typists or by spending too much time looking at notes to try to
  remember what they meant to type.

\item Instructors can spend too much time typing in boilerplate code
  that is needed by the lesson, but not directly relevant to it (such
  as library import statements). Not only does this slow things down, it can distract learners from the
  intended thrust of a lesson.  As Willingham~\cite{willingham-book} says,
  ``Memory is the residue of thought''; if the instructor spends their
  time typing boilerplate, that may be all that learners take away.
  This can be avoided by starting with a partial skeleton that
  includes the boilerplate, or having it on hand to copy and paste
  when needed.  (Of the two, we prefer the former, since learners
  may not be able to keep up with copying and pasting.)

\end{enumerate}

\rulemajor{Have students make predictions}{predictions}

When instructors are using live coding, they usually run the program
several times during its development to show what it does.  Surprising research
from peer instruction in physics education shows that learners who
observe a demonstration \textit{do not learn better} than those who
did not see the demonstration~\cite{crouch-demo}, and in fact many
learners misremember the outcome of demonstrations
afterwards~\cite{miller-demo}.  In other words, demonstrations can
actually be useless or actively harmful.

The key to making demonstrations more effective is
to make learners predict the outcome of the demonstration before performing it.
Crucially, their prediction should be in some way recorded or public,
e.g. by a show of hands,
by holding up a cue card (A/B/C/D),
or by talking to their neighbor.
We speculate that the sting of being publicly wrong leads learners to pay more attention
and to reflect on what they are learning;
regardless of whether this hypothesis is true,
instructors should be careful not to punish or criticise students who predicted wrongly,
but rather to use those incorrect predictions as a spur to further exploration and explanation.

\rulemajor{Use pair programming}{pair-programming}

Pair programming is a software development practice
in which two programmers share one computer.
One person (called the driver) does the typing,
while the other (called the navigator) offers comments and suggestions.
The two switch roles several times per hour.
Pair programming is a good practice in real-life programming \cite{hannay},
and also a good way to teach \cite{mcdowell}.
Partners can not only help each other out during practical exercises,
but can also clarify each other's misconceptions when the solution is presented.

Both parties involved in pair programming learn while doing it.
The weaker gets individual instruction from the stronger,
while the stronger learns by explaining,
and by being forced to reconsider things which they may not have thought about in a while.
When pair programming is used it is important to put everyone in pairs,
not just the learners who may be struggling,
so that no one feels singled out.
It's also important to have people switch roles within each pair three or four times per hour,
so that the stronger personality in each pair does not dominate the session.

\rulemajor{Use worked examples with labelled subgoals}{worked-examples}

Learning to program involves learning the syntax and semantics of a programming language,
but also involves learning how to construct programs.
A good way to guide students through constructing programs is the use of worked examples:
step-by-step guides showing how to solve an existing problem.

Instructors usually provide many similar programming examples for learners to practice on.
But since learners are novices,
they may not see the similarity between examples:
finding the highest rainfall from a list of numbers and
finding the first surname alphabetically from a list of names
may seem like quite different problems to learners,
even though more advanced programmers would recognise them as isomorphic.

Morrison et al.~\cite{morrison-subgoals,morrison-parsons,margulieux} have shown that students perform better
when worked examples are broken down into steps (or subgoals)
which are given names (or labels) -- an example is given in Figure~\ref{fig:subgoal}.
Subgoal labels provide a structure which allow learners to see the similarities between coding problems and to communicate with their peers and instructors more efficiently.  Learners can then apply the labels to future tasks that they attempt themselves.

\rulemajor{Stick to one language}{one-language}

A principle that applies across all areas of education is that
transference only comes with mastery \cite{gick-holyoak}.
Courses should therefore stick to one language until learners have progressed far enough with it
to be able to distinguish the forest from the trees.
While an experienced programmer can,
for example,
take what they know about loops and function calls in one language
and re-use that understanding in a language with a different syntax or semantics,
a newcomer does not yet know which elements of their knowledge are central
and which are accidental.
Attempting to force transference too early---e.g.,
requiring them to switch from Python to JavaScript in order to do a web programming course
early in their education---will confuse learners and erode their confidence.

\rulemajor{Use authentic tasks}{authentic-tasks}

Guzdial et al.\ found that having learners manipulate images, audio, and video
in their early programming assignments
increased retention in two senses:
learners remembered more of the material when re-tested after a delay,
and were more likely to stay in computing programs~\cite{guzdial-media}.
This is a particular instance of a larger observation:
learners find authentic tasks more engaging than abstracted examples.

A classic question in computing (and mathematics) education
is whether problems are better with context (e.g., find the highest student grade)
or without (e.g. find the maximum of the list of numbers).
Bouvier et al.~\cite{bouvier-context} examined this with a multi-university study
and found no difference between the two.
They suggest that since it makes no difference,
other considerations (such as motivation) should be primary.

One caution about choosing context is that context can inadvertently exclude some people while drawing others in.
For example,
many educators use computer games as a motivating example for programming classes,
but some learners may associate them with violence and racial or gender stereotypes,
or simply find them unenjoyable.
Whatever examples are chosen,
the goal must be to move learners as quickly as possible
from ``hard and boring'' to ``easy and exciting'' \cite{repenning}.

To help students accomplish a visible and satisfting result
quickly, instructors can provide some pre-written software libraries or source code that start students
closer to the end goal.  The idea that students must start from scratch and write all the code they need
themselves is the relic of a bygone era of home microcomputers (and it was not true even then).
Pick the task that you actually want to the students to engage in, and provide everything else pre-made.

\rulemajor{Remember that novices are not experts}{not-experts}

This principle is tautological, but it is easily forgotten.
Novices program differently than experts~\cite{parnin-expertise},
and need different approaches or tools.
If you ask a professional programmer to iterate over a list of integers and produce the average,
they can write the code within seconds,
using stored knowledge of the exact pattern required.
A novice will approach this problem totally differently:
they need to remember the syntax for the different parts,
they need to know how to iterate over a list,
how to use an accumulator variable,
and so on.

Novices may need to spend time thinking about an algorithm on paper
(something expert programmers rarely need,
as they have usually memorised most common algorithmic patterns).
They may need to construct examples in guided steps.
They may struggle to debug.
Debugging usually involves contrasting what is happening to what should be happening,
but a novice's grasp on what should be happening is usually fragile.

Novices do not become professionals simply by doing what professionals do at a slower pace.
We do not teach reading by taking a classic novel and simply proceeding more slowly.
We teach by using shorter books with simpler words and larger print.
So in programming,
we must take care to use small, self-contained tasks at a level suitable for novices,
with tools that suit their needs, and without scoffing.

\rulemajor{Don't just code}{not-just-code}

Our final rule for teaching programming is that you don't have to program to do it.
Faced with the challenges of learning syntax, semantics, algorithms, and design,
examples that seem small to instructors can still easily overwhelm novices.
Breaking the problem down into smaller single-concept pieces
can reduce the cognitive load to something manageable.

For example,
a growing number of educators are including Parsons Problems
in their pedagogic repertoire~\cite{parsons,morrison-parsons}.
Rather than writing programs from scratch,
learners are given the lines of code they need to solve a problem,
but in jumbled order.
Re-ordering them to solve the problem correctly
allows them to concentrate on mastering control flow
without having to devote mental energy to recalling syntax
or the specifics of library functions.  They are also liked by learners; Ericson et al.~\cite{ericson-ebook} found
that learners were more likely to attempt Parsons Problems than nearby multiple choice questions in an ebook.

\section*{Conclusion}

The ten rules presented here are backed up by scientific research.
Like any research involving human subjects,
studies of computing education must necessarily be hedged with qualifiers.
However,
we do know a great deal,
and are learning more each year.
Venues like SIGCSE (\url{http://sigcse.org/}), ITiCSE (\url{http://iticse.acm.org/}) and ICER (\url{https://icer.hosting.acm.org})
present a growing number of rigorous, insightful studies
with immediate practical application.
Future work may overturn or qualify some of our ten simple rules,
but they form a solid basis for any educational effort to the best of our current knowledge.

We offer one final observation: do not forget the human element.
Programmers have a reputation for pouring scorn on certain programming languages (e.g., PHP),
or for gatekeeping (e.g., stating that you cannot learn programming if you did not start young).
If you are teaching someone to program,
the last thing you want to do is make them feel like they can't succeed
or that any existing skill they have (no matter when or how acquired) is worthless.
Make your learners feel that they can be a programmer,
and they just might become one.

\section*{Acknowledgments}

We are grateful to Barbara Ericson for her comments on a draft of this article.

\pagebreak

\begin{figure}[h]
\begin{mdframed}
\begin{verbatim}
for (int i = 1; i < 10; i++) {
    if (i < 3 || i >= 8) {
        System.out.println("Yes");
    }
}
\end{verbatim}
How many times will the above code print out the word Yes?\\
a) 10\\
b) 5\\
c) 4\\
d) 3
\end{mdframed}
\caption{An example multiple choice question probing learners' understanding of loops and integer comparisons.}
\label{fig:mcq}
\end{figure}

\begin{figure}[h]
\begin{mdframed}
\noindent
\textbf{Conventional Materials}
\begin{verbatim}
1. Click on "My Blocks" to see the blocks for components you created.
2. Click on "clap"
3. Drag out a when clap.Touched block
4. Click on “clapSound”
5. Drag out call clapSound.Play
6. Connect it after when clap.Touched
\end{verbatim}

\noindent
\textbf{Subgoal Labeled Materials}
\begin{verbatim}
Handle Events from My Blocks

1. Click on "My Blocks" to see the blocks for components you created.
2. Click on "clap"
3. Drag out a when clap.Touched block

Set Output from My Blocks

4. Click on "clapSound" and
5. Drag out call clapSound.Play
6. Connect it after when clap.Touched
\end{verbatim}
\end{mdframed}
\caption{An example of subgoal labeling (taken from \cite{margulieux})}
\label{fig:subgoal}
\end{figure}

% \nolinenumbers

% Either type in your references using
% \begin{thebibliography}{}
% \bibitem{}
% Text
% \end{thebibliography}
%
% or
%
% Compile your BiBTeX database using our plos2015.bst
% style file and paste the contents of your .bbl file
% here. See http://journals.plos.org/plosone/s/latex for
% step-by-step instructions.

\pagebreak

\bibliography{paper}

\end{document}
